% PLANTILLA APA7
% Creado por: Isaac Palma Medina
% Última actualización: 25/07/2021
% @COPYLEFT

% Fuentes consultadas (todos los derechos reservados):  
% Normas APA. (2019). Guía Normas APA. https://normas-apa.org/wp-content/uploads/Guia-Normas-APA-7ma-edicion.pdf
% Tecnológico de Costa Rica [Richmond]. (2020, 16 abril). LaTeX desde cero con Overleaf (1 de 3) [Vídeo]. YouTube. https://www.youtube.com/watch?v=kM1KvHVuaTY Weiss, D. (2021). 
% Formatting documents in APA style (7th Edition) with the apa7 LATEX class. https://ctan.math.washington.edu/tex-archive/macros/latex/contrib/apa7/apa7.pdf @COPYLEFT

%+-+-+-+-++-+-+-+-+-+-+-+-+-++-+-+-+-+-+-+-+-+-+-+-+-+-+-+-+-+-++-+-+-+-+-+-+-+-+-+

% Preámbulo
\documentclass[stu, 12pt, letterpaper, donotrepeattitle, floatsintext, natbib]{apa7}
\usepackage[utf8]{inputenc}
\usepackage{comment}
\usepackage{marvosym}
\usepackage{graphicx}
\usepackage{float}
\usepackage[normalem]{ulem}
\usepackage[spanish]{babel} 
\selectlanguage{spanish}
\useunder{\uline}{\ul}{}
\newcommand{\myparagraph}[1]{\paragraph{#1}\mbox{}\\}

% Portada
\thispagestyle{empty}
\title{\Large Título del documento}
\author{Ing. Luis Mercado} % (autores separados, consultar al docente)
% Manera oficial de colocar los autores:
%\author{Autor(a) I, Autor(a) II, Autor(a) III, Autor(a) X}
\authorsaffiliations{Universidad de los Andes}
\course{Código del curso: Nombre del curso}
\professor{Prof. Francisco Palm}
\duedate{Febrero, 2024}
\begin{document}
\maketitle


% Índices
\pagenumbering{roman}

% Contenido
\renewcommand\contentsname{\large{Índice}}
\tableofcontents
\setcounter{tocdepth}{2}
\newpage

% Fíguras
\renewcommand{\listfigurename}{\large{Índice de fíguras}}
\listoffigures
\newpage

% Tablas
\renewcommand{\listtablename}{\large{Índice de tablas}}
\listoftables
\newpage

% Cuerpo
\pagenumbering{arabic}

\section{\large Título I}
\noindent \maskCitet{cervantes1999}\\
En un lugar de la Mancha, de cuyo nombre no quiero acordarme,
no ha mucho tiempo que vivía un hidalgo de los de lanza en astillero,
adarga antigua, rocín flaco y galgo corredor.

\subsection{Título II}
Una olla de algo más vaca que carnero, salpicón las más noches, duelos 
y quebrantos los sábados, lantejas los viernes, algún palomino de añadidura 
los domingos, consumían las tres partes de su hacienda.

\subsubsection{Título III}
El resto della concluían sayo de velarte, calzas de velludo para las 
fiestas, con sus pantuflos de lo mesmo, y los días de entresemana 
se honraba con su vellorí de lo más fino.

\paragraph{Título IV}
Tenía en su casa una ama que pasaba de los cuarenta, y una sobrina 
que no llegaba a los veinte, y un mozo de campo y plaza, que así 
ensillaba el rocín como tomaba la podadera.

\myparagraph{Título IV ii}
Frisaba la edad de nuestro hidalgo con los cincuenta años; 
era de complexión recia, seco de carnes, enjuto de rostro, gran 
madrugador y amigo de la caza.

\subparagraph{Título V}
Quieren decir que tenía el sobrenombre de Quijada, 
o Quesada, que en esto hay alguna diferencia en los autores 
que deste caso escriben; aunque por conjeturas verosímiles 
se deja entender que se llamaba Quijana.

1.	Planteamiento del Problema	
2.	Objetivos de la Investigación	
Objetivo general	
Objetivos específicos	
3.	Justificación de la investigación	8
4.	Antecedentes de la Investigación	
5.	Metodología de la Investigación	15
5.1 Tipo de investigación	
5.2 Diseño de la investigación	
5.3 Fuente de datos	
5.4 Población y Muestra	
5.5 Técnicas de Procesamiento y Análisis de la Información	
5.6 Camino Metodológico	
6.	Cronograma de Actividades	
Referencias Bibliográficas	

Planteamiento del problema

La determinación de distancias estelares se presenta como un 
desafío interesante que puede abarcar tanto la estadística 
como la astronomía. Este proyecto pretende abordar la 
complejidad surgida al medir las distancias en el espacio, 
lo cual representa un problema fundamental para la astronomía, 
como por ejemplo para la comprensión de la estructura y evolución 
de nuestra galaxia.

Para calcular las distancias estelares, se emplea la paralaje, 
un ángulo formado por una estrella con la Tierra desde dos puntos 
diferentes en su órbita. Sin embargo, la medición precisa de la paralaje 
se ve afectada por diversos errores, como lo sería el movimiento propio 
de las estrellas y la extinción interestelar. La estadística bayesiana 
es una herramienta flexible y creativa, ofrece la posibilidad de modelar 
las relaciones entre variables, teniendo en cuenta factores que pueden 
corregir estos errores.

En lugar de abordar solo la paralaje, este proyecto propone utilizar 
el diagrama color-magnitud, que relaciona las magnitudes absolutas y 
luminosidades estelares. La ecuación  m = M + 5*(log10() +1) establece 
la conexión con la distancia, donde m es la magnitud aparente, M es la 
magnitud absoluta y d es la distancia. La estadística bayesiana se 
aplicará para corregir las estimaciones de paralaje, aprovechando 
la información adicional proporcionada por el diagrama color-magnitud.

Aunque ya se han explorado enfoques similares en trabajos anteriores, 
como los de Lauren Anderson, et al (2018) y Zhang, Green, y Rix (2023), 
este proyecto se distingue por proponer un enfoque completamente bayesiano, 
similar al presentado por Leistedt y Hogg (2017), y proporcionar documentación 
específica de los casos analizados.

Hipótesis

La implementación de un modelo bayesiano jerárquico usando el diagrama 
color-magnitud puede mejorar significativamente la precisión de las 
estimaciones de distancias estelares.

Objetivos mínimos.

Desarrollo de un modelo bayesiano jerárquico usando el diagrama color-magnitud para la estimación de distancias estelares.
Evaluar la precisión del modelo adaptado.
Documentar la metodología y el modelado estadístico.

Justificación de la investigación

Este proyecto sobrepasa los límites de la física y se establece como una 
aportación relevante al ámbito de la estadística. Pues al aplicar y adaptar 
métodos bayesianos para corregir errores en mediciones astronómicas, 
se busca no solo validar la robustez de esta metodología sino también 
explorar su flexibilidad y creatividad en un contexto estadístico.

La relevancia de este enfoque radica en su capacidad para impactar diversas 
disciplinas, incluida la geografía y la ingeniería, al proporcionar nuevas 
perspectivas sobre la corrección de mediciones en entornos no terrestres. 
Además, al ser un proyecto de licenciatura, se enfatiza la oportunidad de 
contribuir a la comunidad académica mediante la aplicación de la estadística 
bayesiana en un contexto específico.

La importancia de explorar y proponer nuevos modelos estadísticos se destaca, 
ya que la astronomía, en constante evolución, presenta continuamente nuevos 
desafíos y oportunidades para el análisis de datos. En este sentido, se hace 
especial mención a la disponibilidad de datos recientes de misiones como Gaia 
de la Agencia Espacial Europea, lo que proporciona una base sólida para la 
implementación y evaluación de modelos bayesianos en el contexto estadístico.

Metodología

En función de lo anteriormente explicado según Cox y Donnaly (2011) podemos 
clasificar este trabajo como un estudio observacional. Y se pueden abarcar 
varios aspectos como la formulación teórica y la revisión bibliográfica al 
problema de estimación de paralajes. En líneas generales, la investigación 
documentará y propondrá un modelo bayesiano jerárquico para optimizar las 
estimaciones de distancia a partir de los datos de Gaia DR3 mediante el 
empleo del diagrama color-magnitud.

Siendo más específico se plantea los siguientes pasos:

Revisión bibliográfica sobre los modelos jerárquicos bayesianos y otros métodos para la estimación de distancias estelares.
Desarrollo e implementación de un modelo jerárquico bayesiano para la estimación de distancias estelares.

Validación del modelo.

Discusión de las implicaciones.

Referencias bibliográficas
% l Leistedt, B., & Hogg, D. W. (2017). Hierarchical probabilistic inference of the color–magnitude diagram and shrinkage of stellar distance uncertainties. The Astronomical Journal, 154(6), 222.
% l Gaia Collaboration (2016). Gaia data release 1. Astronomy & Astrophysics, 595, A1.
% l Hogg, D. W., & Foreman-Mackey, D. (2018). Data analysis of astronomical objects: A bayesian approach (2nd ed.). Cambridge University Press.
% l Bailer-Jones, C. A. (2015). Estimating distances from parallaxes. Publications of the Astronomical Society of the Pacific, 127(956), 9942
% l Lauren Anderson, et al (2018). Improving Gaia Parallax Precision with a Data-driven Model of Stars. The Astronomical Journal. AJ 156 145.
% l Edenhofer, G., Zucker, C., Frank, P., Saydjari, A.K., Speagle, J.S., Finkbeiner, D.P., & Enßlin, T. (2023). A Parsec-Scale Galactic 3D Dust Map out to 1.25 kpc from the Sun.

Recursos necesarios para llevar a efecto la investigación.

Acceso a la base de datos de Gaia DR3.
Herramientas de programación para el desarrollo del modelo bayesiano.
Recursos computacionales para el análisis de datos.
Bibliografía relevante sobre modelos bayesianos y técnicas de análisis de datos astronómicos.
Como nota, todos estos recursos están disponibles de forma gratuita vía Internet.

\newpage
% Referencias
\renewcommand\refname{\large\textbf{Referencias bibliográficas}}
% \bibliography{bibliography.bib}

\end{document}

