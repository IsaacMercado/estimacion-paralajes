% PLANTILLA APA7
% Creado por: Isaac Palma Medina
% Última actualización: 25/07/2021
% @COPYLEFT

% Fuentes consultadas (todos los derechos reservados):  
% Normas APA. (2019). Guía Normas APA. https://normas-apa.org/wp-content/uploads/Guia-Normas-APA-7ma-edicion.pdf
% Tecnológico de Costa Rica [Richmond]. (2020, 16 abril). LaTeX desde cero con Overleaf (1 de 3) [Vídeo]. YouTube. https://www.youtube.com/watch?v=kM1KvHVuaTY Weiss, D. (2021). 
% Formatting documents in APA style (7th Edition) with the apa7 LATEX class. https://ctan.math.washington.edu/tex-archive/macros/latex/contrib/apa7/apa7.pdf @COPYLEFT

%+-+-+-+-++-+-+-+-+-+-+-+-+-++-+-+-+-+-+-+-+-+-+-+-+-+-+-+-+-+-++-+-+-+-+-+-+-+-+-+

% Preámbulo
\documentclass[stu, 12pt, letterpaper, donotrepeattitle, floatsintext, natbib]{apa7}
\usepackage[utf8]{inputenc}
\usepackage{comment}
\usepackage{marvosym}
\usepackage{graphicx}
\usepackage{float}
\usepackage[normalem]{ulem}
\usepackage[spanish]{babel}
\usepackage[document]{ragged2e}

\selectlanguage{spanish}
\useunder{\uline}{\ul}{}
\newcommand{\myparagraph}[1]{\paragraph{#1}\mbox{}\\}

% Portada
\thispagestyle{empty}
\title{\Large Título del documento}
\author{Ing. Luis Mercado} % (autores separados, consultar al docente)
% Manera oficial de colocar los autores:
%\author{Autor(a) I, Autor(a) II, Autor(a) III, Autor(a) X}
\authorsaffiliations{Universidad de los Andes}
\course{Código del curso: Nombre del curso}
\professor{Prof. Francisco Palm}
\duedate{Febrero, 2024}
\begin{document}
\maketitle


% Índices
\pagenumbering{roman}

% Contenido
\renewcommand\contentsname{\large{Índice}}
\tableofcontents
\setcounter{tocdepth}{2}
\newpage

% Fíguras
% \renewcommand{\listfigurename}{\large{Índice de fíguras}}
% \listoffigures
% \newpage

% Tablas
% \renewcommand{\listtablename}{\large{Índice de tablas}}
% \listoftables
% \newpage

% Cuerpo
\pagenumbering{arabic}



% \section{\large Título I}
% \subsection{Título II}
% \subsubsection{Título III}
% \paragraph{Título IV}
% \myparagraph{Título V}
% \subparagraph{Título VI}

\subsection{Planteamiento del Problema}

La determinación de distancias estelares se presenta como un
desafío interesante que puede abarcar tanto la estadística
como la astronomía. Este proyecto pretende abordar la
complejidad surgida al medir las distancias en el espacio,
lo cual representa un problema fundamental para la astronomía,
como por ejemplo para la comprensión de la estructura y evolución
de nuestra galaxia.

Para calcular las distancias estelares, se emplea la paralaje,
un ángulo formado por una estrella con la Tierra desde dos puntos
diferentes en su órbita. Sin embargo, la medición precisa de la paralaje
se ve afectada por diversos errores, como lo sería el movimiento propio
de las estrellas y la extinción interestelar. La estadística bayesiana
es una herramienta flexible y creativa, ofrece la posibilidad de modelar
las relaciones entre variables, teniendo en cuenta factores que pueden
corregir estos errores.

En lugar de abordar solo la paralaje, este proyecto propone utilizar
el diagrama color-magnitud, que relaciona las magnitudes absolutas y
luminosidades estelares. La ecuación $\displaystyle M=m+5\left(\log_{10}p+1\right)$ 
establece la conexión con la distancia, donde $m$ es la magnitud aparente, $M$ es la
magnitud absoluta y $d$ es la distancia. La estadística bayesiana se
aplicará para corregir las estimaciones de paralaje, aprovechando
la información adicional proporcionada por el diagrama color-magnitud.

Aunque ya se han explorado enfoques similares en trabajos anteriores,
como los de \maskCitet{Anderson_2017} y \maskCitet{Zhang_2023},
este proyecto se distingue por proponer un enfoque completamente bayesiano,
similar al presentado por \maskCitet{Leistedt_2017}, y proporcionar documentación
específica de los casos analizados.

Hipótesis

La implementación de un modelo bayesiano jerárquico usando el diagrama
color-magnitud puede mejorar significativamente la precisión de las
estimaciones de distancias estelares.

\subsection{Objetivos de la Investigación}

\subsubsection{Objetivo general}

\subsubsection{Objetivos específicos}

\begin{itemize}
    \item Desarrollo de un modelo bayesiano jerárquico usando el diagrama color-magnitud para la estimación de distancias estelares.
    \item Evaluar la precisión del modelo adaptado.
    \item Documentar la metodología y el modelado estadístico.
\end{itemize}

\subsection{Justificación de la investigación}

Este proyecto sobrepasa los límites de la física y se establece como una
aportación relevante al ámbito de la estadística. Pues al aplicar y adaptar
métodos bayesianos para corregir errores en mediciones astronómicas,
se busca no solo validar la robustez de esta metodología, sino también
explorar su flexibilidad y creatividad en un contexto estadístico.

La relevancia de este enfoque radica en su capacidad para impactar diversas
disciplinas, incluida la geografía y la ingeniería, al proporcionar nuevas
perspectivas sobre la corrección de mediciones en entornos no terrestres.
Además, al ser un proyecto de licenciatura, se enfatiza la oportunidad de
contribuir a la comunidad académica mediante la aplicación de la estadística
bayesiana en un contexto específico.

La importancia de explorar y proponer nuevos modelos estadísticos se destaca,
ya que la astronomía, en constante evolución, presenta continuamente nuevos
desafíos y oportunidades para el análisis de datos. En este sentido, se hace
especial mención a la disponibilidad de datos recientes de misiones como Gaia
de la Agencia Espacial Europea, lo que proporciona una base sólida para la
implementación y evaluación de modelos bayesianos en el contexto estadístico.

\subsection{Antecedentes de la Investigación}

\subsection{Metodología de la Investigación}

En función de lo anteriormente explicado según \maskCitet{Cox_Donnelly_2011} podemos
clasificar este trabajo como un estudio observacional. Y se pueden abarcar
varios aspectos como la formulación teórica y la revisión bibliográfica al
problema de estimación de paralajes. En líneas generales, la investigación
documentará y propondrá un modelo bayesiano jerárquico para optimizar las
estimaciones de distancia a partir de los datos de Gaia DR3 mediante el
empleo del diagrama color-magnitud.

Siendo más específico se plantea los siguientes pasos:

\begin{itemize}
    \item Revisión bibliográfica sobre los modelos jerárquicos bayesianos y otros métodos para la estimación de distancias estelares.
    \item Desarrollo e implementación de un modelo jerárquico bayesiano para la estimación de distancias estelares.
    \item Validación del modelo.
    \item Discusión de las implicaciones.
\end{itemize}

\subsubsection{Tipo de investigación}
\subsubsection{Diseño de la investigación}
\subsubsection{Fuente de datos}
\subsubsection{Población y Muestra}
\subsubsection{Técnicas de Procesamiento y Análisis de la Información}

Recursos necesarios para llevar a efecto la investigación.

\begin{itemize}
    \item Acceso a la base de datos de Gaia DR3.
    \item Herramientas de programación para el desarrollo del modelo bayesiano.
    \item Recursos computacionales para el análisis de datos.
    \item Bibliografía relevante sobre modelos bayesianos y técnicas de análisis de datos astronómicos.    
\end{itemize}

\textbf{Como nota}, todos estos recursos están disponibles de forma gratuita vía Internet.

\subsubsection{Camino Metodológico}
\subsection{Cronograma de Actividades}

\newpage
% Referencias
\renewcommand\refname{\large\textbf{Referencias Bibliográficas}}
\bibliography{bibliography}

\end{document}

