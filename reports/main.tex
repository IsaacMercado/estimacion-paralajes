\documentclass[stu, 12pt, letterpaper, donotrepeattitle, floatsintext, natbib]{apa7_ula}
\usepackage[utf8]{inputenc}
\usepackage{comment}
\usepackage{marvosym}
\usepackage{graphicx}
\usepackage{float}
\usepackage{natbib}
\usepackage{pgfgantt}
\usepackage{multirow}
\usepackage[normalem]{ulem}
\usepackage[spanish]{babel}
\usepackage[document]{ragged2e}

\selectlanguage{spanish}

\setlength{\parindent}{1.5cm}

\useunder{\uline}{\ul}{}

% Portada
\thispagestyle{empty}
\title{
    \uppercase{\large Modelo Bayesiano Jerárquico para Mejorar las 
    Estimaciones de Distancia Estelar a través del 
    Diagrama Color-Magnitud}
}
\author{Autor: Ing. Luis Mercado}
\professor{Tutor: Prof. Francisco Palm}
\authorsaffiliations{
    \uppercase{Universidad de los Andes} \\
    \uppercase{Facultad de ciencias económicas y sociales} \\
    \uppercase{Escuela de estadística}
}
\course{
    \uppercase{Propuesta de trabajo de grado} \\
    Para optar al título de Licenciado en Estadística
}
\duedate{
    \uppercase{Mérida, Venezuela} \\ 
    Febrero, 2024
}

\begin{document}
\maketitle

% Índices
\pagenumbering{roman}

% Contenido
\renewcommand\contentsname{\large{Índice}}
\tableofcontents
\setcounter{tocdepth}{2}
\newpage

% Fíguras
% \renewcommand{\listfigurename}{\large{Índice de fíguras}}
% \listoffigures
% \newpage

% Tablas
% \renewcommand{\listtablename}{\large{Índice de tablas}}
% \listoftables
% \newpage

% Cuerpo
\pagenumbering{arabic}

% \section{\large Título I}
% \subsection{Título II}
% \subsubsection{Título III}
% \paragraph{Título IV}
% \subparagraph{Título VI}

\subsection{Planteamiento del Problema}

\justifying \parindent=1.5cm

La determinación de distancias estelares se presenta como un desafío
interesante que puede abarcar tanto la estadística como la astronomía.
Este proyecto pretende abordar la complejidad surgida al medir las
distancias en el espacio, lo cual representa un problema fundamental
para la astronomía, como por ejemplo para la comprensión de la estructura
y evolución de nuestra galaxia.

Para calcular las distancias estelares, se emplea la paralaje,
un ángulo formado por una estrella con la Tierra desde dos puntos
diferentes en su órbita. Sin embargo, la medición precisa de la paralaje
se ve afectada por diversos errores, como lo sería el movimiento propio
de las estrellas y la extinción interestelar. La estadística bayesiana
es una herramienta flexible y creativa, ofrece la posibilidad de modelar
las relaciones entre variables, teniendo en cuenta factores que pueden
corregir estos errores.

En lugar de abordar solo la paralaje, este proyecto propone utilizar
el diagrama color-magnitud, que relaciona las magnitudes absolutas y
luminosidades estelares. La ecuación $\displaystyle M=m+5\left(\log_{10}d+1\right)$
establece la conexión con la distancia, donde $m$ es la magnitud aparente, $M$ es la
magnitud absoluta y $d$ es la distancia. La estadística bayesiana se
aplicará para corregir las estimaciones de paralaje, aprovechando
la información adicional proporcionada por el diagrama color-magnitud.

Aunque ya se han explorado enfoques similares en trabajos anteriores,
como los de \citet{Anderson_2017} y \citet{Zhang_2023},
este proyecto se distingue por proponer un enfoque completamente bayesiano,
similar al presentado por \citet{Leistedt_2017}, y proporcionar documentación
específica de los casos analizados.

% Hipótesis:
% La implementación de un modelo bayesiano jerárquico usando el diagrama
% color-magnitud puede mejorar significativamente la precisión de las
% estimaciones de distancias estelares.

\subsection{Objetivos de la Investigación}

\subsubsection{Objetivo general}

Desarrollar de un modelo bayesiano jerárquico usando el
diagrama color-magnitud para la estimación de distancias estelares.

\subsubsection{Objetivos específicos}

\begin{itemize}
    \item Revisar la literatura sobre modelos bayesianos para la estimación de distancias estelares.
    \item Desarrollar un modelo bayesiano jerárquico para la estimación de distancias estelares.
    \item Evaluar la precisión del modelo adaptado a los datos de Gaia DR3.
    \item Documentar la metodología y el modelado estadístico.
\end{itemize}

\subsection{Justificación de la investigación}

Este proyecto sobrepasa los límites de la física y se establece como una
aportación relevante al ámbito de la estadística. Pues al aplicar y adaptar
métodos bayesianos para corregir errores en mediciones astronómicas,
se busca no solo validar la robustez de esta metodología, sino también
explorar su flexibilidad y creatividad en un contexto estadístico.

La relevancia de este enfoque radica en su capacidad para impactar diversas
disciplinas, incluida la geografía y la ingeniería, al proporcionar nuevas
perspectivas sobre la corrección de mediciones en entornos no terrestres.
Además, al ser un proyecto de licenciatura, se enfatiza la oportunidad de
contribuir a la comunidad académica mediante la aplicación de la estadística
bayesiana en un contexto específico.

La importancia de explorar y proponer nuevos modelos estadísticos se destaca,
ya que la astronomía, en constante evolución, presenta continuamente nuevos
desafíos y oportunidades para el análisis de datos. En este sentido, se hace
especial mención a la disponibilidad de datos recientes de misiones como Gaia
de la Agencia Espacial Europea, lo que proporciona una base sólida para la
implementación y evaluación de modelos bayesianos en el contexto estadístico.

\subsection{Antecedentes de la Investigación}

Actualmente, contamos con una amplia gama de modelos estadísticos para
estimar distancias estelares, pero pocas de estas técnicas usan métodos bayesianos,
motivo por el cual se encuentran pocos trabajos que aborden este tema en
la comunidad internacional, y menos aún en la comunidad hispanohablante. Sin embargo,
se pueden encontrar algunos como lo son
\citet{Leistedt_2017}, \citet{Anderson_2017} y \citet{Zhang_2023}.

El trabajo más acorde a la propuesta de este proyecto es el de \citet{Leistedt_2017},
quienes presentan un modelo bayesiano jerárquico para la estimación de distancias
estelares, utilizando mediciones de paralaje y magnitudes aparentes. En este trabajo
se usaron datos de TGAS, que son datos combinados de la misión Gaia y el catálogo de
estrellas de Tycho-2, dando como resultado una mejora significativa en la precisión
de las estimaciones de distancias estelares.

Otro trabajo con un enfoque similar es el de \citet{Anderson_2017}, quienes resuelven
el problema de la estimación de distancias estelares mediante el uso de un modelo
bayesiano empírico con el método de devolución extrema XD, también usando TGAS con buenos
resultados, pero en la actualidad se reconoce que los métodos empíricos no son tan
robustos como los modelos jerárquicos completamente bayesianos, aunque si mucho más
rápidos.

Por último, el trabajo de \citet{Zhang_2023}, que no solo provee correcciones de las
de distancias estelares, sino que también estimaciones de masas y edades,
utilizando un enfoque similar a \citet{Anderson_2017}, es decir, usan un
modelo bayesiano empírico, usando datos de Gaia DR2. Por lo que es
en la actualidad el trabajo más actualizado en este campo.

\subsection{Metodología de la Investigación}

\subsubsection{Tipo de investigación}

En función de lo anteriormente explicado según \citet{Cox_Donnelly_2011} podemos
clasificar este trabajo como un estudio observacional, puesto que no tenemos
control de las variables de estudio. Y va a abarcar
varios aspectos como la formulación teórica y la revisión bibliográfica al
problema de estimación de paralajes. En líneas generales, la investigación
documentará y propondrá un modelo bayesiano jerárquico para optimizar las
estimaciones de distancia a partir de los datos de Gaia DR3 mediante el
empleo del diagrama color-magnitud.

\subsubsection{Diseño de la investigación}

Referente al diseño de la investigación, se plantea usar el flujo de trabajo
bayesiano propuesto por \citet{gelman2020bayesian}, que especifica los pasos
o mejor dicho la ruta a seguir para el desarrollo de un modelo
bayesiano.

\subsubsection{Fuente de datos}

Este estudio se basa en la información recopilada por la misión
Gaia de la Agencia Espacial Europea. La misión Gaia es un proyecto
de cartografía estelar que tiene como objetivo medir las posiciones,
movimientos y propiedades físicas de más de mil millones de estrellas
en la Vía Láctea y más allá. La tercera versión de datos de Gaia (Gaia DR3)
proporciona mediciones precisas de paralaje y magnitudes aparentes para más
de mil millones de estrellas, para más información puede ver \citet{Vallenari2022GaiaDR}.

También se considerarán datos de otras fuentes, como el catálogo de estrellas
de Hipparcos y el catálogo de estrellas de Tycho-2.

\subsubsection{Población y Muestra}

Para motivos de este estudio, se considerará una muestra completamente
aleatoria de estrellas de la base de datos de Gaia DR3.
También hay que tener en cuenta que como se
está elaborando un estudio bayesiano, siguiendo las recomendaciones de
\citet{gelman2020bayesian} se utilizará las simulaciones de un modelo teórico
dispuesto por el mismo proyecto Gaia, para la validación del modelo.

\subsubsection{Técnicas de Procesamiento y Análisis de la Información}

En este trabajo se plantea la estimación de un número $N$ de
estrellas tomando en cuenta su relación con el diagrama color-magnitud
por lo tanto se plantea el uso de un modelo bayesiano jerárquico.

Siendo más específicos, se tratará de seguir el enfoque de \citet{Leistedt_2017}
usando un modelo generativo explicado por \citet[p, 11]{gelman2020bayesian}
que permita definir una distribución prior para la distancia, otra
para el diagrama color-magnitud y en la función de verosimilitud que
explique la relación de los datos con los priors.

Ahora bien, para la extracción de datos se usará el lenguaje de programación python
con la librería astroquery que me permite extraer los datos y los modelos se definirán
usando la librería pymc, la cual está especializada y optimizada para el proceso
de desarrollo, validación y selección de modelos bayesianos, contando con un gran
cantidad de ejemplos y tutoriales disponibles.

\subsubsection{Camino Metodológico}

A grandes rasgos la investigación requerirá los siguientes pasos:

\begin{itemize}
    \item Revisión bibliográfica sobre los modelos jerárquicos bayesianos y otros métodos para la estimación de distancias estelares.
    \item Desarrollo e implementación de un modelo jerárquico bayesiano para la estimación de distancias estelares.
    \item Validación del modelo.
    % \item Documentación.
\end{itemize}

Se debe tener en cuenta que según la descripción de
\citet{gelman2020bayesian}, el desarrollo e implementación de un
modelo jerárquico bayesiano es un proceso iterativo que se empieza
con un modelo inicial simple y luego poco a poco se van agregando
variaciones y/o correcciones tratando de encontrar el modelo
adecuado para el problema planteado.

\subsection{Cronograma de Actividades}

\begin{ganttchart}[
        vgrid,
        hgrid,
        % bar label font=\mdseries\small\color{black!70},
        y unit title=1cm,
        y unit chart=0.8cm,
        bar label node/.append style={left=2cm},
        bar label font=\raggedright\mdseries\small,
        group label node/.append style={left=.6cm}
    ]{1}{12}
    \gantttitle{2024}{12} \\
    \gantttitle{Abril}{4} \gantttitle{Mayo}{4} \gantttitle{Junio}{4} \\
    \gantttitlelist{1,...,4}{1} \gantttitlelist{1,...,4}{1} \gantttitlelist{1,...,4}{1} \\

    \ganttgroup{Revisión bibliográfica}{1}{2} \\
    \ganttbar{Cálculo de distancias estelares}{1}{2} \\
    \ganttbar{Modelos bayesianos en astronomía}{1}{2} \\

    \ganttgroup{Formulación del anteproyecto}{3}{4} \\
    \ganttbar{Planteamiento del problema}{3}{4} \\
    \ganttbar{Objetivos}{3}{4} \\
    \ganttbar{Justificación}{3}{4} \\
    \ganttbar{Antecedentes}{3}{4} \\
    \ganttbar{Marco metodológico}{3}{4} \\

    \ganttgroup{Desarrollo del modelo bayesiano}{5}{7} \\
    \ganttbar{Definición del modelo inicial}{5}{7} \\
    \ganttbar{Ajuste (Fit) del modelo}{5}{7} \\
    \ganttbar{Validación del modelo}{5}{7} \\
    \ganttbar{Corrección del modelo}{5}{7} \\

    \ganttgroup{Redacción del proyecto de investigación}{8}{9} \\
    \ganttbar{Marco teórico}{8}{9} \\
    \ganttbar{Análisis e interpretación de resultados}{8}{9} \\
    \ganttbar{Conclusiones y recomendaciones}{8}{9} \\

    \ganttgroup{Presentación y revisión}{10}{12} \\
    \ganttbar{Revisión del proyecto con el tutor}{10}{10} \\
    \ganttbar{Presentación de informe final}{11}{11} \\
    \ganttbar{Exposición de los resultados}{12}{12}
\end{ganttchart}



\newpage
% Referencias
\renewcommand\refname{\large\textbf{Referencias Bibliográficas}}
\bibliography{bibliography}

\end{document}

